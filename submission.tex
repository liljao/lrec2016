\documentclass[11pt,a4paper]{article}
\usepackage[T1]{fontenc}
\usepackage[utf8]{inputenc}
\usepackage{lrec2006}
%\usepackage{apacite}
\usepackage{covington}
\usepackage{tikz-qtree}
\usepackage{tikz-dependency}
\usepackage{ctable}


\usepackage{relsize}
%\bibliographystyle{apacite}

\let\w=\emph
\let\tag=\textsf
\let\app=\textsc
\newcommand\deprel[1]{\\\textsc{#1}}

\tikzset{edge from parent/.style={->,draw,font={\scshape\small}},
    every tree node/.style={align=center,anchor=north},
    level distance=10ex}

\setlength\textfloatsep{3.5mm}

\title{Norwegian Universal Dependencies}
%\title{Universal Dependencies for Norwegian}

\author{\bf Lilja {\O}vrelid\\[0.2ex]
         \mbox{}Department of Informatics, University of Oslo\\[0.2ex]
         \mbox{}{\tt liljao@ifi.uio.no}\\[0.2ex]
        \vspace*{1.5ex}}
        

\date{}

\frenchspacing

\begin{document}
\maketitle

\section{Introduction}
%323
With the increasing popularity of dependency-based representations of
syntactic structure in recent years, a wealth of different dependency
annotation schemes have surfaced. It has been shown that the choice of
dependency scheme influences parsing results \cite{Sch:Abe:Rap:12} as
well as down-stream applications \cite{Elm:Joh:Kle:13} and even though
attempts have been made to contrast different schemes theoretically
\cite{Iva:Oep:Ovr:12}, it is clear that the diversity of
representation makes comparisons difficult. Cross-linguistically even
more so, and it can often be difficult to tease apart
aspects of annotation scheme from typological differences in cross-lingual learning \cite{Soe:11,Skj:Ovr:12}.

Universal Dependencies (UD) \cite{Mar:Doz:Sil:14,Nivre:15} is a recent
community-driven effort to create cross-linguistically consistent
syntactic annotation. UD is based on the Stanford dependency scheme
\cite{Mar:Mac:Man:06} which has become a widely used dependency scheme
for English in recent years.  A number of existing dependency
treebanks have been converted to UD \cite{Pyy:Kan:Miss:15,Nivre:14}
and new data has also been annotated from scratch in order to enable
multilingual parser development, cross-lingual learning and
typological studies of syntactic structure. Treebanks involved in this
effort represent a diverse range of languages such as English, German,
Swedish, Spanish, Italian, Persian, Japanese, and the UD release 1.1
contains treebanks for as many as 34 different languages of varying
sizes. 

This paper describes a fully automatic conversion procedure for the
Norwegian Dependency Treebank (NDT) to UD. Due to differences both in
the tag set, as well as structural analyses, the conversion requires
non-trivial transformations of the dependency trees, in addition to
mappings of tags and labels that make reference to a combination of
various kinds of linguistic information.  This paper details the
mapping of PoS tags, morphological features and dependency relations
and provides a description of the structural changes made to NDT
analyses in order to make it compliant with the UD guidelines. The full
converted treebank will be made available with the next release of the
UD treebanks scheduled for November 2015.


\section{NDT and UD}
%138
Universal Dependencies (UD) builds on several previous initiatives for universally common
morphological \cite{Zem:08,Pet:Das:McD:12} and syntactic dependency \cite{McD:Niv:Qui:13,Ros:Mas:Mar:14}
annotation. Among its main tenets are the primacy of content-words,
i.e. content words, as opposed to function words, are syntactic heads
wherever possible. It is intended to be a universal annotation scheme,
i.e. applicable to any language, however also offers some
possibilities for language-specific information.

The Norwegian Dependency Treebank (NDT) \cite{Sol:Skj:Ovr:14} contains
morphological and syntactic annotation for both varieties of written
Norwegian (Bokm{\aa}l and Nynorsk). The morphological annotation
follows the Oslo-Bergen Tagger scheme \cite{Hag:Joh:Nok:00}.  The
syntactic annotation scheme is, to a large extent, based on the
Norwegian Reference Grammar \cite{Faa:Lie:Van:97}.  and the dependency
representations are inspired by choices made in comparable treebanks,
in particular the Swedish treebank Talbanken
\cite{Niv:Nil:Hal:2006}.

%45
%% \ctable[botcap,
%%     caption={Annotation choices in the NDT},
%%     label=tb:choices,
%% ]{ll}{}{
%%         \FL
%% \multicolumn{1}{c@{\hspace{1em}}}{Head} & \multicolumn{1}{c@{\hspace{0.5em}}}{Dependent} \ML
%% Finite verb & Complementizer \NN
%% Finite auxiliary & Lexical verb \NN
%% Infinitival marker & Lexical verb \NN
%% Preposition & Prepositional complement \NN
%% Noun & Determiner\NN
%% First conjunct & Subsequent conjuncts 
%%         \LL
%%     }

\section{Parts-of-speech}
% 78 + 287 = 365
The part-of-speech tag set used in the UD scheme is based on the
Universal PoS tag set of \newcite{Pet:Das:McD:12} and contains 17 tags. The NDT tag set
contains 19 tags.  The conversion of the part-of-speech information in
NDT to the UD pos tag set is fairly straightforward and largely relies
on a direct mapping presented in Table \ref{tb:pos}.  A few
parts-of-speech require conversion rules which make reference to
additional information in the treebank, represented by disjunction
in the mapping. Below we will discuss a few of these cases.

%122
\ctable[botcap,
    caption={Mapping between NDT and UD parts-of-speech},
    label=tb:pos,
]{ll}{}{
        \FL
NDT & UD\ML
{\tt adj} & {\tt ADJ}\NN
{\tt adv} & {\tt ADV}\NN
{\tt clb} & {\tt PUNCT}, {\tt SYM}\NN
{\tt det} & {\tt DET}, {\tt NUM}\NN
{\tt konj} & {\tt CONJ}\NN
{\tt interj} & {\tt INTJ}\NN
{\tt inf-merke} & {\tt PART}\NN
{\tt prep} & {\tt ADP}, {\tt ADV}\NN
{\tt pron} & {\tt PRON}\NN
{\tt <komma>} & {\tt PUNCT}\NN
{\tt sbu} & {\tt SCONJ}\NN
{\tt <strek>} & {\tt PUNCT}\NN
{\tt subst} & {\tt NOUN}, {\tt PROPN}\NN
{\tt <anf>} & {\tt PUNCT}\NN
{\tt <parentes-slutt>} & {\tt PUNCT}\NN
{\tt <parentes-beg>} & {\tt PUNCT}\NN
{\tt symb} & {\tt SYM}\NN
{\tt ukjent} & {\tt X}\NN
{\tt verb} & {\tt AUX}, {\tt VERB}
        \LL
    }

The universal scheme makes a distinction between proper and common
nouns at the part-of-speech level. This information can be found among
the morphological features in NDT ({\tt prop}), hence the mapping is
straightforward.

For verbs UD distinguishes auxiliaries ({\tt AUX}) from main
verbs ({\tt VERB}).  This distinction is not explicitly made in
NDT, hence our conversion procedure must make use of the syntactic
structure of the verbs in order to implement this distinction. Verbs
that have a direct, non-finite dependent (a dependent with the NDT
dependency relation {\sc inf}) are marked as auxiliaries and all other
verbs as regular verbs.

\section{Morphological information}
%109
In addition to part-of-speech information, NDT contains a rich
inventory of morphological features, e.g. information about properties
like gender, definiteness, tense, voice, etc. The UD guidelines
specify a universal set of morphological features and the conversion
between the two does not require reference to
information in addition to the feature information. The feature
mapping is described in Table \ref{tb:feats}.  Note that since the
number of UD features is larger than the NDT features, some of the NDT
features correspond to a set of UD features, e.g. the NDT features for
verbs ({\tt pres, pret}) which instantiate both the {\tt Mood, Tense}
and {\tt VerbForm} features.

%109
\ctable[botcap,
    caption={Mapping between NDT and UD morphological features},
    label=tb:feats,
]{lp{4cm}}{}{
        \FL
NDT & UD\ML
{\tt mask,fem,n{\o}yt} & {\tt Gender=Masc,Fem,Neut}\NN
{\tt ent,fl} & {\tt Number=Sing,Plur}\NN
{\tt be,ub} & {\tt Definite=Def,Ind}\NN
{\tt pres,pret} & {\tt Mood=Ind, Tense=Pres,Past, VerbForm=Fin}\NN
{\tt perf-part} & {\tt VerbForm=Part}\NN
{\tt imp} & {\tt Mood=Imp, VerbForm=Fin}\NN
{\tt pass} & {\tt Voice=Pass}\NN
{\tt inf} & {\tt VerbForm=Inf}\NN
{\tt 1,2,3} & {\tt Person=1,2,3}\NN
{\tt nom,akk,gen} & {\tt Case=Nom,Acc,Gen}\NN
{\tt pos,komp,sup} & {\tt Degree=Pos,Cmp,Sup}\NN
{\tt hum} & {\tt Animacy=Anim}\NN
{\tt pers} & {\tt PronType=Prs}\NN
{\tt dem} & {\tt PronType=Dem}\NN
{\tt sp} & {\tt PronType=Int}\NN
{\tt res} & {\tt PronType=Rcp}\NN
{\tt poss} & {\tt Poss=Yes}\NN
{\tt refl} & {\tt Refl=Yes}
        \LL
    }

\section{Structural conversion}
%63
The NDT annotation scheme differs structurally from the UD scheme in
a number of important ways.  The conversion is therefore non-trivial
and requires a set of structural rules which operate on the dependency
graphs in addition to a mapping procedure over the dependency labels.
The conversion is implemented as a cascade of structural rules followed by a
relation conversion procedure over the modified graph structures.
The structural rules employ a small set of graph operations that reverse, reattach, delete and add arcs.

\subsection{Verbal groups}
%106
NDT consistently marks the finite verb as head of a clause, with other
non-finite verbs as dependents ({\tt INFV}), see example
(\ref{ex:verb}a). In a parallel manner, infinitival markers are also
annotated as heads with the infinitival verb as its dependent, see
(\ref{ex:infmark}a). UD on the other hand annotates the lexical, main
verb as head of the verbal group and various finite and non-finite
auxiliaries receive an auxiliary relation ({\tt aux, auxpass}) , see
(\ref{ex:verb}b) and (\ref{ex:infmark}b) below.  The conversion rule
locates the main verb within the chain of nonfinite dependents of the
finite verb and makes this node the head of the other verbs in the
chain.

%134
\begin{examples}
\item\label{ex:verb}
\begin{minipage}[t]{3cm}
a)
    \begin{dependency}[arc edge, text only label, theme=simple]
        \begin{deptext}[column sep=.2cm]
            skal \& ha \& sett \\
            \w{shall} \& \w{have} \& \w{seen}\\
        \end{deptext}
    \depedge{1}{2}{\large\sc infv}
    \depedge{2}{3}{\large\sc infv}
    \deproot[edge unit distance=1.5ex]{1}{\large\sc }
    \end{dependency}
\end{minipage}
\hspace{0.5cm}
\begin{minipage}[t]{3cm}
b)
%\item\label{ex:ndt-verb}
    \begin{dependency}[arc edge, text only label, theme=simple]
        \begin{deptext}[column sep=.5cm]
            skal \& ha \& sett \\
            \w{shall} \& \w{have} \& \w{seen}\\
        \end{deptext}
    \depedge{3}{1}{\large aux}
    \depedge{3}{2}{\large aux}
    \deproot[edge unit distance=1.5ex]{3}{\large }
    \end{dependency}
\end{minipage}
\item\label{ex:infmark}
\begin{minipage}[t]{3cm}
a)
    \begin{dependency}[arc edge, text only label, theme=simple]
        \begin{deptext}[column sep=.2cm]
            {\aa} \& ha \& sett \\
            \w{to} \& \w{have} \& \w{seen}\\
        \end{deptext}
    \depedge{1}{2}{\large\sc infv}
    \depedge{2}{3}{\large\sc infv}
    \deproot[edge unit distance=1.5ex]{1}{}
    \end{dependency}
\end{minipage}
\hspace{0.5cm}
\begin{minipage}[t]{3cm}
b)
    \begin{dependency}[arc edge, text only label, theme=simple]
        \begin{deptext}[column sep=.5cm]
            {\aa} \& ha \& sett \\
            \w{to} \& \w{have} \& \w{seen}\\
        \end{deptext}
    \depedge{3}{1}{\large mark}
    \depedge{3}{2}{\large aux}
    \deproot[edge unit distance=1.5ex]{3}{}
    \end{dependency}
\end{minipage}
\end{examples}


\subsection{Copula constructions}
%42
The treatment of copula constructions within the UD scheme differs
markedly from that of the NDT by appointing the predicative element as
head of the entire construction and attaching the copula verb with a
special relation {\tt cop}, see (\ref{ex:cop}b). Our conversion thus
reverses the arc between the copula and its complement and reattaches
all its dependents to the predicative element.
%68
\begin{examples}
\item\label{ex:cop}
\begin{minipage}[t]{3cm}
a)
    \begin{dependency}[arc edge, text only label, theme=simple]
        \begin{deptext}[column sep=.2cm]
            de \& er \& fattige \\
            \w{they} \& \w{are} \& \w{poor}\\
        \end{deptext}
    \depedge{2}{1}{\large\sc subj}
    \depedge{2}{3}{\large\sc spred}
    \deproot[edge unit distance=1.5ex]{2}{\large\sc finv}
    \end{dependency}
\end{minipage}
\hspace{0.5cm}
\begin{minipage}[t]{3cm}
b)
%\item\label{ex:ndt-verb}
    \begin{dependency}[arc edge, text only label, theme=simple]
        \begin{deptext}[column sep=.4cm]
            de \& er \& fattige\\
            \w{they} \& \w{are} \& \w{poor}\\
        \end{deptext}
    \depedge{3}{1}{\large nsubj}
    \depedge{3}{2}{\large cop}
    \deproot[edge unit distance=1.5ex]{3}{\large root}
    \end{dependency}
\end{minipage}
\end{examples}


\subsection{Prepositions and their complements}
%47
In NDT, prepositions are heads of their prepositional complements
which receive the {\tt PUTFYLL} label, see (\ref{ex:prep}a). Seeing
that languages differ greatly in their use of pre/postpositions, the
UD scheme annotates the prepositional complement as head and attaches
the preposition using the {\tt case} relation, see (\ref{ex:prep}b). In conversion this is once again a matter of reversing the arc and reattaching dependents to the new head.
%56
\begin{examples}
\item\label{ex:prep}
\begin{minipage}[t]{2.5cm}
a)
    \begin{dependency}[arc edge, text only label, theme=simple]
        \begin{deptext}[column sep=.2cm]
            p{\aa} \& menyen \\
            \w{on} \& \w{menu-the} \\
        \end{deptext}
    \depedge{1}{2}{\large\sc putfyll}
    \deproot[edge unit distance=1.5ex]{1}{}
    \end{dependency}
\end{minipage}
\hspace{0.5cm}
\begin{minipage}[t]{2.5cm}
b)
    \begin{dependency}[arc edge, text only label, theme=simple]
        \begin{deptext}[column sep=.2cm]
            p{\aa} \& menyen \\
            \w{on} \& \w{menu-the} \\
        \end{deptext}
    \depedge{2}{1}{\large case}
    \deproot[edge unit distance=1.5ex]{2}{}
    \end{dependency}
\end{minipage}
\end{examples}


\subsection{Predicatives}
%119
NDT distinguishes several types of predicatives, both predicatives
that are arguments of verbs (subject predicative {\tt SPRED} and
object predicative {\tt OPRED}) and ``free predicatives'' which are
not arguments of the verb, but nonetheless characterize either a
subject or an object in the preceding context (free subject
predicative {\tt FSPRED} and free object predicative {\tt
  FOPRED}). Both of these are attached to the finite verb in NDT, as we can see in (\ref{ex:pred}a). In a
similar manner, UD distinguishes between obligatory and optional
predicatives, where the former are analyzed using the {\tt xcomp}
relation and attached to the main predicate, whereas the optional
predicatives are attached as adverbial clauses ({\tt acl}) modifying the argument
they characterize, see (\ref{ex:pred}b). Our conversion thus attaches the {\tt FSPRED} argument to its sibling subject argument, {\tt FOPRED} to an object sibling.
%65
\begin{examples}
\item\label{ex:pred}
\begin{itemize}
\item[(a)]
    \begin{dependency}[arc edge, text only label, theme=simple]
        \begin{deptext}[column sep=.2cm]
          taler \& Solstad \& uredd \\
          \w{speaks} \& \w{Solstad} \& \w{unscared} \\
        \end{deptext}
            \depedge{1}{2}{\large\sc subj}
            \depedge{1}{3}{\large\sc fspred}
            \deproot[edge unit distance=1.5ex]{1}{}
    \end{dependency}
\item[(b)]
    \begin{dependency}[arc edge, text only label, theme=simple]
        \begin{deptext}[column sep=.2cm]
          taler \& Solstad \& uredd \\
          \w{speaks} \& \w{Solstad} \& \w{unscared} \\
        \end{deptext}
            \depedge{1}{2}{\large nsubj}
            \depedge{2}{3}{\large acl}
            \deproot[edge unit distance=1.5ex]{1}{}
    \end{dependency}
\end{itemize}
\end{examples}


\subsection{Coordination}
%65
Coordination is a phenomenon which exhibits considerable variation in terms
of dependency representation across various annotation schemes \cite{Pop:Mar:Ste:13}.
As we can see from the example in (\ref{ex:coord}), the analyses
chosen in the NDT and UD schemes are fairly similar in their choice of
the first conjunct as head of the coordinate structure. They differ
mainly in the attachment of the conjunction and the relation names.
%88
\begin{example}
\label{ex:coord}
\begin{itemize}
\item[(a)]
    \begin{dependency}[arc edge, text only label, theme=simple]
        \begin{deptext}[column sep=.2cm]
          kamskjell \& , \& piggvar \& og \& lammefilet \\
          \w{scallops} \& , \& \w{turbot} \& \w{and} \& \w{lambfilet}\\
        \end{deptext}
            \depedge{1}{2}{\large\sc ik}
            \depedge{1}{3}{\large\sc koord}
            \depedge{1}{5}{\large\sc koord}
            \depedge{5}{4}{\large\sc konj}
            \deproot[edge unit distance=1.5ex]{1}{}
    \end{dependency}
\item[(b)]
    \begin{dependency}[arc edge, text only label, theme=simple]
        \begin{deptext}[column sep=.2cm]
          kamskjell \& , \& piggvar \& og \& lammefilet \\
          \w{scallops} \& , \& \w{turbot} \& \w{and} \& \w{lambfilet}\\
        \end{deptext}
            \depedge{1}{2}{\large punct}
            \depedge{1}{3}{\large conj}
            \depedge{1}{4}{\large cc}
            \depedge{1}{5}{\large conj}
            \deproot[edge unit distance=1.5ex]{1}{}
    \end{dependency}
\end{itemize}
%\item\label{ex:ndt-verb}
\end{example}

\section{Conversion of dependency relations}
%105
A minority of the dependency relations in NDT may be converted
directly, based on the mapping described in Table \ref{tb:dep}, the
rest require the formulation of mapping constraints which make
reference to information in addition to the dependency relation
itself, i.e. part-of-speech tag, morphological features, dependency
structure or even a combination of these. An overview of
the non-direct mapping of dependency relations that require additional
information from the linguistic context is provided in Table \ref{tb:multidep}.
These mappings often also require heuristics which approximate some
syntactic property which is not explicitly annotated in NDT. Below we
will present these heuristics and their usage in the conversion.

%102
\ctable[botcap,
    caption={Direct mapping between NDT and UD dependency relations.},
    label=tb:dep,
]{ll}{}{
        \FL
NDT & UD\ML
{\tt APP} & {\tt appos}\NN
{\tt FSUBJ} & {\tt expl}\NN
{\tt FOBJ} & {\tt expl}\NN
{\tt FSPRED} & {\tt acl}\NN
{\tt FOPRED} & {\tt acl}\NN
{\tt FRAG} & {\tt root}\NN
{\tt IOBJ} & {\tt iobj}\NN
{\tt OPRED} & {\tt xcomp}\NN
{\tt INTERJ} & {\tt discourse}\NN
{\tt KONJ} & {\tt cc}\NN
{\tt KOORD} & {\tt conj}\NN
{\tt KOORD-ELL} & {\tt remnant}\NN
{\tt IP} & {\tt punct}\NN
{\tt IK} & {\tt punct}\NN
{\tt PAR} & {\tt parataxis}\NN
{\tt SPRED} & {\tt xcomp}\NN
{\tt UKJENT} & {\tt goeswith}
        \LL
    }


%118
\paragraph{Active/Passive}There are two ways of expressing passive voice in Norwegian: a
morphological passive expressed by an addition of a {\it -s} to the
verb, e.g. {\it danses} 'to be danced' or a periphrastic passive which
is composed of the auxiliary {\it bli} 'to become' and a participle
form, e.g. {\it danset} 'danced'. Only the morphological passive is
marked explicitly as being in the passive voice.  We therefore define
a heuristic which counts a lexical main verb as passive, if it is (i)
a morphological passive, or (ii) is a participle headed by a form of
the auxiliary {\it bli}.  This heuristic is used in the conversion of
passive auxiliaries {\tt auxpass} and passive subjects {\tt nsubjpass,
  csubjpass}.

%128
\paragraph{Nominal/Clausal}Several of the UD relations assume a distinction between nominal and
clausal elements. This distinction is complicated somewhat by the fact
that in copula constructions, as described above, the complement of
the copula is head of the construction as a whole. This means that
adjectives or even nouns may be counted as clausal in contexts where
they have a copula dependent, as in (\ref{ex:cop}b).  In the
conversion we introduce a notion of a \emph{predicate}, which may be
either verbal ({\tt AUX, VERB}) or the complement in a copula
construction. This notion is used to distinguish nominal and clausal
subjects ({\tt nsubj, nsubjpass} vs. {\tt csubj, csubjpass}), objects
({\tt dobj} vs. {\tt ccomp, xcomp}), various modifiers ({\tt nmod}
vs. {\tt acl}) and adverbials ({\tt nmod} vs. {\tt advcl}).
%TO-DO: adverb subjects?? footnote

%59
\paragraph{Control}UD is inspired by the syntactic framework of Lexical Functional
Grammar \cite{Kap:Bre:82} and adopts its distinction between complement
clauses with obligatory subject control ({\tt xcomp}) and those without
({\tt ccomp}). The notion of control is not native to NDT, hence we
approximate it by requiring the presence of an explicit subject dependent of the
head verb of the clause.


%45
\paragraph{Negation}UD distinguishes negation modifiers which modify either a noun ({\it
  no problem}) or a predicate (in the aforementioned sense) ({\it is
  not a problem, doesn't argue}). Our conversion explicitly marks the
negative determiner ({\it ingen} 'no') and the negative adverb ({\it
  ikke} 'not') in Norwegian.

%141
\paragraph{Particles}NDT distinguishes between transitive and intransitive prepositions, or
so-called particles, in the annotation. In order to account for the
relation between the verb and its particle we introduce the
language-specific relation {\tt compound:prt}, for prepositions which
are attached to a verb and furthermore does not have an explicit prepositional complement,
e.g. {\it si opp} 'discontinue' (lit. 'say up').  

\paragraph{Relative clauses}
In many languages, relative markers are pronouns. In Norwegian, there
are good reasons for choosing to treat the relative marker {\it som}
'that' as a subjunction \cite{Faa:Lie:Van:97}. This relative marker,
unlike those found in many other languages does not inflect (as in
English {\it who-whom}, or German {\it der-die-das}) and exclusively
occurs initially in a subordinate clause. In our conversion we
introduce a language-specific variant of the clausal relation {\tt
  acl}, {\tt acl:relcl} which signals that this is a relative clause.
%% The relative marker {\it som} 'that' receives an argument role based
%% on a heuristic which checks for the presence or absence of other
%% argument siblings.
%69
%% \begin{example}
%% \label{ex:relclause}
%% \begin{minipage}[t]{3cm}
%% a)
%%     \begin{dependency}[arc edge, text only label, theme=simple]
%%         \begin{deptext}[column sep=.3cm]
%%           dem \& som \& lider \\
%%           \w{those} \& \w{who} \& \w{suffer} \\
%%         \end{deptext}
%%             \depedge{1}{3}{\large\sc atr}
%%             \depedge{3}{2}{\large\sc sbu}
%%             \deproot[edge unit distance=1.5ex]{1}{}
%%     \end{dependency}
%% \end{minipage}
%% \hspace{0.5cm}
%% \begin{minipage}[t]{3cm}
%% b)
%%     \begin{dependency}[arc edge, text only label, theme=simple]
%%         \begin{deptext}[column sep=.3cm]
%%           dem \& som \& lider \\
%%           \w{those} \& \w{who} \& \w{suffer} \\
%%         \end{deptext}
%%             \depedge{1}{3}{\large acl:relcl}
%%             \depedge{3}{2}{\large nsubj}
%%             \deproot[edge unit distance=1.5ex]{1}{}
%%     \end{dependency}
%% \end{minipage}
%% \end{example}

%% \subsection{Numerical modifiers}
%% UD distinguishes numerical modifiers of nouns, e.g. {\it 42,
%%   fifty}. These are annotated as regular determiners in NDT, but are
%% part of a larger group of quantifiers which receive the additional
%% feature {\tt kvant}, e.g. {\it noen} 'some', {\it ingen} 'no'. In
%% order to distinguish the numerical modifiers from the other
%% quantifying determiners, we use a regular expression which recognizes
%% digits and numerical expressions like {\it trettito} 'thirty-two'.
%99
\ctable[botcap,
    caption={Non-direct mapping between NDT and UD dependency relations; requires additional constraints with reference to PoS, morphological features or dependency context.},
    label=tb:multidep, 
]{lp{4cm}l}{}{
        \FL
NDT & UD \ML
{\tt ADV} & {\tt advcl, advmod, compound:prt, neg, nmod} \NN
{\tt ATR} & {\tt acl:relcl, amod, nmod} \NN
{\tt DET} & {\tt nmod, nummod, det} \NN
{\tt FINV} & {\tt aux, auxpass, root} \NN
{\tt FLAT} & {\tt foreign, name} \NN
{\tt OBJ, POBJ} & {\tt dobj, ccomp, xcomp} \NN
{\tt SBU} & {\tt nsubj, nsubjpass, dobj, iobj, mark} \NN
{\tt SUBJ, PSUBJ} & {\tt nsubj, nsubjpass, csubj, csubjpass} 
        \LL
    }


%TO-DO som is sconj, mark relation follows
%TO-DO change code to treat relcl
%TO-DO change description in paper, can still have acl:relcl


%% preposition stranding with particle analysis, correct correlate dobj:
%% som henholdsvis Sveits og Norge har lagt ut

%% preposition stranding, incorrect correlate dobj:
%% en tid som vi er relativt godt fornøyd med
%% et søppellag som vi ikke skal tape mot (wrong particle analysis)
%% de store sakene som han arbeider med

%% a different problem with analysis, "double correlate''.., not correct dobj analysis
%% hva som foregikk i Norge på det aktuelle tidspunktet
%% hva som blir hans nye klubb
%% hvilke tiltak som kan bidra

%% to prinsipper som begge er viktige ?? what is correlate 

%% no way of distinguishing relative clauses without relativizer (possible if correlate is something other than subject)..

%% if decide to skip relclause analysis, should not treat som as pronoun!! Check NRG for argumentation, evidence that is not pronoun. What about Swedish??

%% NRG:
%% - not a pronoun in Norwegian, not conjugated (ubøyelig) and only occurs as initial element in subordinate clause

%% - if so, should be 'mark'

%TO-DO: UD relations not used: mwe, etc. check

%% %Include this? Not quite sure about this argument. Is this a particle or not? tests..
%% Another side effect of our treatment of particles is that not all intransitive prepositions are actually particles.
%% One such case is caused by so-called \emph{preposition stranding}, where the complement of the preposition occurs elsewhere in the sentence, as in the relative clause in (\ref{ex:prep-strand}).
%% \begin{examples}
%% \item\label{ex:prep-strand}
%% \gll en applikasjon de kan være stolte av
%% an application they can be proud of
%% \glt `an application they can be proud of'
%% \glend
%% \end{examples}

%\clearpage
\smaller
\bibliographystyle{lrec2006}
\bibliography{lrec2016}

\end{document}


% From structural conversion
%% \subsection{Root}
%% %52
%% In NDT, there is no designated root label. Rather, the root of the
%% dependency graph may have different labels, e.g. {\tt FINV} (finite
%% verb) or {\tt FRAG} (fragment), depending on the structure of the
%% sentence. In the conversion, every node with the dummy node (0) as
%% head receives the {\tt root} relation.

%From PoS conversion

%% The relative pronoun {\it som} 'who/that' is analyzed as a subjunction
%% in NDT, whereas the universal scheme, and thus our conversion, uses
%% the pronominal tag {\tt PRON} for these words.

%% NDT explicitly marks headings using the {\tt |} symbol, tagged as a
%% clause boundary ({\tt clb}) along with information in the
%% morphological features {\tt <overskrift>}. These are converted to the
%% UD symbol tag {\tt SYM}.

%% Numerical expressions, e.g. {\it to} 'two' or {\it 45} are tagged as
%% determiners in NDT, also when they do not explicitly modify a nominal,
%% e.g. {\it de to gjorde en imponerende innsats} 'the two did an
%% impressive job'. These are however marked as 'kvant'
%% (quantificational) in the morphological features, hence the conversion
%% mapping makes reference to this property in order to distinguish {\tt NUM} from {\tt DET}.

%% NDT implements a somehwat broader definition of prepositions than the UD
%% scheme and includes several cases which are counted as adverbs in the
%% universal schemes. In particular, this is the case for demonstrative
%% adverbs such as {\it her} 'here' and {\it der} 'there'. These are
%% followingly converted to adverbs ({\tt ADV}) in the PoS conversion, whereas all
%% other prepositions are converted to {\tt ADP}.

%possibly change?


%from dependency relation conversion
%% On manual inspection of the
%% converted data we find that this conversion also gives us the
%% preposition in fixed expressions which require the combination of a
%% verb, direct object and a preposition, such as {\it snu ryggen til}
%% 'turn (ones) back to' as well as certain fixed combinations of
%% prepositions which function as a whole syntactically, e.g. {\it til og
%%   med} 'even (lit. to and with)'. Without any existing annotation
%% distinguishing these from the regular particle constructions these are
%% difficult to differentiate automatically.  


%% \subsection{Relative clauses}
%% The analyses of relative clauses differ notably between the two
%% annotation schemes. Both schemes treat the relative clause as a
%% clausal modifier of a nominal element ({\tt ATR} vs {\tt acl}),
%% however the treatment of the relative marker/pronoun differs. NDT
%% treats relative markers as subordinating conjunctions which depend on
%% the finite verb of the relative clause,\footnote{They argue for this
%%   by pointing to the fact that relative markers unlike many other
%%   languages do not conjugate (who-whom, der-die-das) and only occur
%%   initially in a subordinate clause \cite{Faa:Lie:Van:97}} UD treats
%% the relative marker as a pronoun which occupies an argument relation
%% in the relative clause. In our conversion we introduce a
%% language-specific variant of the clausal relation {\tt acl}, {\tt
%%   acl:relcl} which signals that this is a relative clause. The
%% relative marker {\it som} 'that' receives an argument role based on a
%% heuristic which checks for the presence or absence of other argument
%% siblings.
